%
% Copyright (c) 2015 Radoslaw Kujawa.
% All rights reserved.
%
% Redistribution and use in source and binary forms, with or without
% modification, are permitted provided that the following conditions
% are met:
%
% 1. Redistributions of source code must retain the above copyright
%    notice, this list of conditions and the following disclaimer.
% 2. Redistributions in binary form must reproduce the above copyright
%    notice, this list of conditions and the following disclaimer in the
%    documentation and/or other materials provided with the distribution.
%
% THIS SOFTWARE IS PROVIDED BY RADOSLAW KUJAWA (THE AUTHOR) AND CONTRIBUTORS
% ``AS IS'' AND ANY EXPRESS OR IMPLIED WARRANTIES, INCLUDING, BUT NOT LIMITED
% TO, THE IMPLIED WARRANTIES OF MERCHANTABILITY AND FITNESS FOR A PARTICULAR
% PURPOSE ARE DISCLAIMED.  IN NO EVENT SHALL THE AUTHOR OR CONTRIBUTORS
% BE LIABLE FOR ANY DIRECT, INDIRECT, INCIDENTAL, SPECIAL, EXEMPLARY, OR
% CONSEQUENTIAL DAMAGES (INCLUDING, BUT NOT LIMITED TO, PROCUREMENT OF
% SUBSTITUTE GOODS OR SERVICES; LOSS OF USE, DATA, OR PROFITS; OR BUSINESS
% INTERRUPTION) HOWEVER CAUSED AND ON ANY THEORY OF LIABILITY, WHETHER IN
% CONTRACT, STRICT LIABILITY, OR TORT (INCLUDING NEGLIGENCE OR OTHERWISE)
% ARISING IN ANY WAY OUT OF THE USE OF THIS SOFTWARE, EVEN IF ADVISED OF THE
% POSSIBILITY OF SUCH DAMAGE.
%
% 
\documentclass[dvipsnames,table]{beamer}
\usepackage{polski}

\usetheme{Rochester}
\usecolortheme{orchid}

\usepackage{listings}
\usepackage{ucs}
\usepackage[utf8x]{inputenc}
\usepackage{wasysym}
\usepackage[normalem]{ulem}
\usepackage{amsmath}
\usepackage{hyperref}
\usepackage{tikzsymbols}

\setbeamertemplate{navigation symbols}{}
\setbeamertemplate{caption}[numbered]
\setbeamerfont{caption}{size=\scriptsize}
\setbeamercolor{framenote}{bg=OSEC-red!25}
\setbeamercolor{rednote}{bg=Red!25}
\setbeamercolor{palette primary}{use=structure,fg=white,bg=OSEC-red}
\setbeamercolor{palette secondary}{use=structure,fg=white,bg=OSEC-red2}

\setbeamertemplate{itemize item}{\scriptsize\raise1pt\hbox{\donotcoloroutermaths$\blacktriangleright$}}
\setbeamertemplate{itemize subitem}{\tiny\raise1pt\hbox{\donotcoloroutermaths$\bullet$}}
\setbeamertemplate{itemize subsubitem}{\tiny\raise1pt\hbox{\donotcoloroutermaths{--}}}

\setbeamertemplate{enumerate item}{\insertenumlabel.}
\setbeamertemplate{enumerate subitem}{\insertenumlabel.\insertsubenumlabel}
\setbeamertemplate{enumerate subsubitem}{\insertenumlabel.\insertsubenumlabel.\insertsubsubenumlabel}
\setbeamertemplate{enumerate mini template}{\insertenumlabel}

\setbeamercolor{itemize item}{fg=OSEC-red, bg=OSEC-red}
\setbeamercolor{itemize subitem}{fg=OSEC-red, bg=OSEC-red}
\setbeamercolor{itemize subsubitem}{fg=OSEC-red, bg=OSEC-red}

\setbeamercolor{section number projected}{fg=white,bg=OSEC-red}
\setbeamercolor{subsection number projected}{fg=white,bg=OSEC-red}
\setbeamercolor{button}{bg=OSEC-red,fg=white}

\setbeamertemplate{section in toc}[circle]
\setbeamertemplate{subsection in toc}[square]


\definecolor{OSEC-red}{RGB}{160,29,44}
\definecolor{OSEC-red2}{RGB}{177,76,12}
\hypersetup{colorlinks=true,linkcolor=white,urlcolor=OSEC-red}

\setlength{\tabcolsep}{8pt}
\renewcommand{\arraystretch}{1.2}

\newcommand{\nbsdcolor}[1] {
	{\color{OSEC-red} #1}
}

\lstset{
   language=java,
   basicstyle=\tiny,
   breaklines=true,
   escapechar=\@,
   commentstyle=\color{OSEC-red}
}

\AtBeginSection[]{
\frame{

\begin{center}

{\usebeamerfont{section name}\usebeamercolor[fg]{section name}Część~\insertsectionnumber}
    \vskip1em\par

	\begin{beamercolorbox}[sep=12pt,center]{palette primary}
		\usebeamerfont{section title}\insertsection\par
	\end{beamercolorbox}
\end{center}

%\sectionpage
}
}

\title{Wykorzystanie rekordów DNS \\ do weryfikacji kluczy publicznych SSH}


\author{Radosław Kujawa -- radoslaw.kujawa@osec.pl}

\institute{OSEC}

\begin{document}

\begin{frame}
\titlepage
\end{frame}

\begin{frame}[allowframebreaks]
\frametitle{Spis treści}
{
\hypersetup{colorlinks=true,linkcolor=black,urlcolor=OSEC-red}
\tableofcontents
}
\end{frame}


\section{Zagrożenia bezpieczeństwa związane z weryfikacją kluczy publicznych hostów}

\begin{frame}
\frametitle{Pierwsze połączenie}
\begin{itemize}
	\item Każdy z nas używa SSH...
	\item Ale czy każdy z nas zdaję sobie sprawę z zagrożeń? 
\end{itemize}
\end{frame}

\begin{frame}
\frametitle{Problem - weryfikacja kluczy}
\begin{itemize}
	\item Foo {\tt bar}
	\item \href{http://example.com/}{example}
\end{itemize}
\end{frame}

\begin{frame}[fragile]
\frametitle{Akceptacja klucza publicznego zdalnego hosta}
\scriptsize
\begin{verbatim}
$ ssh example.com
The authenticity of host 'example.com (1.2.3.4)' can't be established.
RSA key fingerprint is 12:75:dd:41:22:3f:6f:88:f7:bc:4e:ad:2e:b5:da:2d.
Are you sure you want to continue connecting (yes/no)?
\end{verbatim}
\end{frame}

\begin{frame}
\frametitle{Gdy klucz zostanie zaakceptowany}
\begin{itemize}
	\item Jest zapisywany w {\tt \$HOME/.ssh/.known\_hosts}
	\item Używany do weryfikacji każdego kolejnego połączenia do tego samego hosta
\end{itemize}
\end{frame}

\section{Jak wykorzystać rekordy SSHFP do zautomatyzowanej weryfikacji kluczy}

\begin{frame}
\frametitle{Mechanizm SSHFP}
\begin{itemize}
	\item Klient inicjuje połączenie z serwerem SSH
	\item Serwer SSH legitymuje się swoim kluczem publicznym
	\item Klient pobiera wartość rekordów SSHFP dla danego FQDN
	\item Klient weryfikuje odcisk klucza dla danego algorytmu, skoro odcik się zgadza...
	\begin{itemize}
		\item Może automatycznie zaakceptować klucz
		\item Może zapytać użytkownika czy klucz zaakceptować, podając informację o zgodności z rekordem SSHFP
	\end{itemize}
	\item Dalej połączenie SSH przebiega jak zazwyczaj
\end{itemize}
\end{frame}

\begin{frame}
\frametitle{Generowanie rekordów SSHFP}
\begin{itemize}
	\item Nie trzeba niczego instalować dodatkowo \Smiley
	\item Znane narzędzie {\tt ssh-keygen} posiada tą funkcjonalność.
	\item {\tt ssh-keygen -r `hostname`}
\end{itemize}
\end{frame}

\section{Konfiguracja klienta SSH}

\begin{frame}
\frametitle{Konfiguracja klienta -- OpenSSH}
\begin{itemize}
	\item Nie trzeba niczego instalować dodatkowo \Smiley
	\item Pliki konfiguracyjne klienta:
	\begin{itemize}
		\item {\tt /etc/ssh/ssh\_config}
		\item {\tt \$HOME/.ssh/config}
	\end{itemize}
	\item Opcje:
	\begin{itemize}
		\item {\tt VerifyHostKeyDNS} na wartość {\tt yes} lub {\tt ask}.
	\end{itemize}
\end{itemize}
\end{frame}




\section{Automatyzacja procesu aktualizacji stref}

\begin{frame}
\frametitle{Automatyzacja aktualizacji stref}
\begin{itemize}
	\item Brak standardu \Sadey
	\item Red Hat IPA/FreeIPA
	\item Dynamic DNS -- nsupdate
	\item Puppet/Chef/Ansible...
	\item Generowanie kluczy w Kickstarcie i własny skrypt
\end{itemize}
\end{frame}

\section{Zabezpieczanie DNS przed spoofingiem}

\begin{frame}
\frametitle{DNS a spoofing}
\begin{itemize}
	\item Brak podpisu elektronicznego w standardzie \Sadey
	\begin{itemize}
		\item Podatność DNS na ataki MITM
	\end{itemize}
	\item DNSSEC rozwiązuje ten problem
	\begin{itemize}
		\item Ale kto z Was już go wdrożył?
		\item Ilu dostawców domen go obsługuje?
	\end{itemize}
\end{itemize}
\end{frame}

\begin{frame}
\frametitle{DNS a spoofing c.d.}
\begin{itemize}
	\item Nawet jeśli ,,spoofowalność'' DNS jest problemem...
	\begin{itemize}
		\item Samo wdrożenie SSHFP jest lepsze w porównaniu z brakiem jakiejkolwiek weryfikacji kluczy publicznych hostów.
	\end{itemize}

\end{itemize}
\end{frame}

\begin{frame}
\frametitle{Koniec\ldots}
\begin{center}
\includegraphics[scale=0.5]{img-oseclogo.png}

Dziękuje!

Czy są pytania?

\end{center}

\end{frame}

 
\end{document}
